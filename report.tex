%% Standard start of a latex document
\documentclass[letterpaper,12pt]{article}
%% Always use 12pt - it is much easier to read
%% Things written after '%' sign, are ignored by the latex editor - they are how to introduce comments into your .tex source
%% Anything mathematics related should be put in between '$' signs.

%% Set some names and numbers here so we can use them below
\newcommand{\name}{James Wu}

%%%%%%
%% There is a bit of stuff below which you should not have to change
%%%%%%

%% AMS mathematics packages - they contain many useful fonts and symbols.
\usepackage{amsmath, amsfonts, amssymb, bm}

%% The geometry package changes the margins to use more of the page, I suggest
%% using it because standard latex margins are chosen for articles and letters,
%% not homework.
\usepackage[paper=letterpaper,left=25mm,right=25mm,top=30mm,bottom=30mm]{geometry}
%% For details of how this package work, google the ``latex geometry documentation''.

%% Fancy headers and footers - make the document look nice
\usepackage{fancyhdr} %% for details on how this work, search-engine ``fancyhdr documentation''
\pagestyle{fancy}

\usepackage{graphicx}
\usepackage{adjustbox}

\setlength{\headheight}{15pt}

%% These put horizontal lines between the main text and header and footer.
\renewcommand{\headrulewidth}{0.4pt}
\renewcommand{\footrulewidth}{0.4pt}
%%%

%%%%%%
%% The above stuff is the same as the first template, but now we are starting to prove things, so we'd like to have a
%% good proof environment that gives us nice formatting and a little square at the end.
%% We'd also like a nice Result environment that prints that up nicely too.
%% Thankfully this exists in latex in the amsthm package
\usepackage{amsthm}
\newtheorem*{thm}{Theorem}
%% This creates a new theorem-like environment called "result", that will be titled "Result".
%% See below for examples of how to use this.
%%%%%%
\usepackage{enumitem}
%% This package allows us to make nice ordered lists with numbers, letters or roman numerals

\usepackage{titlesec}
\titlespacing*{\subsection}{0pt}{0pt}{3.0ex}
\titlespacing*{\subsubsection}{0pt}{3.0ex}{0.5ex}

\usepackage[hang,flushmargin]{footmisc}

\setlength{\parindent}{0em}
\setlength{\parskip}{0.5em}

\allowdisplaybreaks

\renewcommand{\arraystretch}{1.4}

\usepackage{empheq}

\newcommand*\wfbox[1]{\fbox{\hspace{0.4em}#1\hspace{0.4em}}}
\numberwithin{table}{section}
\numberwithin{figure}{section}
\numberwithin{equation}{section}

%% Useful commands
\renewcommand*{\qed}{\hfill\ensuremath{\square}}

\newcommand*{\uvec}[1]{\hat{\bm{#1}}}

\newcommand*{\deriv}[2]{\frac{d #1}{d #2}}
\newcommand*{\pderiv}[2]{\frac{\partial #1}{\partial #2}}
\newcommand*{\nderiv}[3]{\frac{d^{#3} #1}{d #2^{#3}}}
\newcommand*{\npderiv}[3]{\frac{\partial^{#3} #1}{\partial #2^{#3}}}
\newcommand*{\divg}[1]{\nabla \cdot \mathbf{#1}}
\newcommand*{\curl}[1]{\nabla \times \mathbf{#1}}

\newcommand*{\abs}[1]{\left| #1 \right|}
\newcommand*{\norm}[1]{\abs{\abs{\mathbf{#1}}}}

\newcommand*{\ev}[1]{\left<#1\right>}

\renewcommand*{\Re}[1]{\text{Re}\left(#1\right)}
\renewcommand*{\Im}[1]{\text{Im}\left(#1\right)}

\newcommand*{\qimg}[2]{\\ \begin{center}\includegraphics[scale=#1]{#2}\end{center}}

\newcommand*{\Arg}[1]{\text{Arg}\left(#1\right)}
\newcommand*{\Log}[1]{\text{Log}\left(#1\right)}
\newcommand*{\Tr}[1]{\text{Tr}\left(#1\right)}

\newcommand*{\ket}[1]{\left|#1\right>}
\newcommand*{\bra}[1]{\left<#1\right|}
\newcommand*{\braket}[2]{\left<#1\right|\left.\!#2\right>}
\newcommand*{\comm}[2]{\left[#1, #2\right]}

% LaTeX
\newcommand{\fig}[2]{\includegraphics[width=#1\textwidth]{img/#2}}
\newcommand{\centerfig}[2]{\begin{center}\includegraphics[width=#1\textwidth]{img/#2}\end{center}}
\def\items#1{\begin{itemize}#1\end{itemize}}
\def\enum#1{\begin{enumerate}#1\end{enumerate}}
\newcommand{\ccaption}[1]{\caption{\textit{#1}}}

% References
\newcommand{\reffig}[1]{\textbf{Figure \ref{#1}}}
\newcommand{\reftab}[1]{\textbf{Table \ref{#1}}}

%%

\begin{document}
\begin{flushleft}

    % TODO title page, etc.
    % TODO header, footer formatting

    \section{Preface}
    % TODO

    \section{Code Usage}
    % TODO

    \section{Morris-Lecar Neuron}
    
    \subsection{Circuit Model}

    % ML intro
    The Morris-Lecar description \cite{ml} models a neuron as having a membrane potential $v$, defined to be the difference in voltage between the inside and outside of the neuron cell. Current flows through potassium and calcium channels in the membrane, labelled as $I_K$ and $I_{Ca}$, respectively. There is also some current arising form other ions, which is collectively totalled as the leak current $I_L$. The combined potassium, calcium, and leak channels each have an effective conductance (or, taking reciprocals, resistance) and potential. Finally, the membrane has a capacitance $C$. \reffig{fig:ml-circuit} depicts a circuit for this model.

    % ML neuron circuit
    \begin{figure}

        \centering
    
        \centerfig{0.6}{ml-circuit.jpg}
        \caption{TODO + cite ML}
    
        \label{fig:ml-circuit}
    
    \end{figure}

    % ML circuit equation
    Consequently, the membrane potential obeys the following first order differential equation:
    \begin{equation}
        C\deriv{v}{t} = I - g_K (v - V_K) - g_{Ca} (v - V_{Ca}) - g_L (v - V_L)
        \label{eqn:ml-circuit}
    \end{equation}

    % ML conductances
    Nominally, the potassium and calcium conductances are non-constant. Rather, they obey the following equations:
    \begin{align}
        g_K &= \bar{g}_K w \\
        g_{Ca} &= \bar{g}_{Ca} m
    \end{align}
    where $\bar{g}_K, \bar{g}_{Ca}$ are constant.
    
    % ML voltage gating variables
    At constant $v$, the parameters $x = w, m$ are governed by first order differential equations of the following form:
    \begin{equation}
        \label{eqn:ml-voltage-gating}
        \deriv{x}{t} = \lambda_x(v) (x_\infty(v) - x)
    \end{equation}
    
    % ML voltage gating vars simplification
    However, the $m$ timescale is much shorter than the $w$ timescale, so that $m \approx m_\infty(v)$ in (\ref{eqn:ml-circuit}). Furthermore, to be consistent with \cite{snm}, we may re-arrange the $x = w$ version of (\ref{eqn:ml-voltage-gating}) to be of the following form:
    \begin{equation}
        \deriv{w}{t} = \alpha(v)(1  - w) - \beta(v)w
    \end{equation}
    where
    \begin{align}
        \alpha(v) &= \frac{1}{2} \phi \cosh{\left(\frac{v - V_3}{2V_4}\right)}\left(1 + \tanh{\left(\frac{v - V_3}{V_4}\right)}\right) \\
        \beta(v) &= \frac{1}{2} \phi \cosh{\left(\frac{v - V_3}{2V_4}\right)}\left(1 - \tanh{\left(\frac{v - V_3}{V_4}\right)}\right)
    \end{align}
    Furthermore, we have
    \begin{equation}
        m_\infty(v) = \frac{1}{2}\left(1 + \tanh{\left(\frac{v - V_1}{V_2}\right)}\right)
    \end{equation}

    \subsection{Morris-Lecar Parameters}

    % Parameters table
    The following parameters from \cite{snm} were used in this project:
    \begin{table}
    
        \centering
    
        \begin{adjustbox}{max width=\textwidth}
    
            \begin{tabular} { | c | c | c | }
                \hline
                Variable & Value & Units \\
                \hline\hline
                $C$ & 20 & $\mu F/cm^2$ \\
                \hline
                $g_L$ & 2.0 & $mS/cm^2$ \\
                \hline
                $\bar{g}_{Ca}$ & 4.4 & $mS/cm^2$ \\
                \hline
                $\bar{g}_K$ & 8 & $mS/cm^2$ \\
                \hline
                $V_L$ & -60 & $mV/cm^2$ \\
                \hline
                $V_{Ca}$ & 120 & $mV/cm^2$ \\
                \hline
                $V_K$ & -84 & $mV/cm^2$ \\
                \hline
                $V_1$ & -1.2 & $mV/cm^2$ \\
                \hline
                $V_2$ & 18.0 & $mV/cm^2$ \\
                \hline
                $V_3$ & 2.0 & $mV/cm^2$ \\
                \hline
                $V_4$ & 30.0 & $mV/cm^2$ \\
                \hline
                $\phi$ & 0.04 & dimensionless \\
                \hline
            \end{tabular}
    
        \end{adjustbox}
    
        \caption{TODO}
        
        \label{tab:2}
    
    \end{table}

    \section{Dynamics}

    \section{Stochastics}

    \section{Interspike Intervals}

    \section{Linearized Model}

    \section{Poincare-Like Maps}

    \section{Patched Model}

    \section{Next Steps}

    \pagebreak

    \bibliographystyle{plain}
    \bibliography{bibliography.bib}

\end{flushleft}
\end{document}
